\documentclass[11pt]{article}
\usepackage[margin=1.5cm]{geometry}
\usepackage{exam}
\usepackage{amsmath}
\usepackage{framed}
\newcommand{\ben}{\begin{enumerate}}
\newcommand{\een}{\end{enumerate}}
\newcommand{\ra}{\rangle}
\newcommand{\la}{\langle}
\newcommand{\bd}{\begin{displaymath}}
\newcommand{\ed}{\end{displaymath}}
\newcommand{\be}{\begin{equation}}
\newcommand{\ee}{\end{equation}}
\newcommand{\ba}{\begin{array}}
\newcommand{\ea}{\end{array}}
\newcommand{\bea}{\begin{eqnarray}}
\newcommand{\eea}{\end{eqnarray}}
\newcommand{\nn}{\nonumber}
\usepackage{changepage}

\begin{document}
\begin{exam}{minitest6}{4}
\begin{center}
\huge
Solid State Physics, Minitest 6\\
\Large
March 31st 2016 \\ 
Good luck! \\[8mm]
\end{center}

\ben
\item {\bf True or false?}
\ben
\item Bands with a higher curvature have a higher effective mass. \emph{(10 pt)}
\item Conductivity of holes is opposite to the conductivity of electrons. \emph{(10 pt)}
\een
\vspace {5mm}
\answerbox{q1}{1cm}{}

\item {\bf Fermi level of a doped semiconductor}\\
Consider a semiconductor with all parameters known, so $m_e, m_h, E_G, E_A, E_D, N_A, N_D$ are all given.
\ben
\item Draw schematically the density of states $G(E)$ for such a semiconductor, denote all  material parameters on the plot. \emph{(10 pt)}

\answerbox{q2}{8cm}{}

\newpage

\item Compute the value of $E_F$ in the high temperature limit, such that the semiconductor behaves intrinsically (but still $kT\ll E_G$). \emph{(15 pt)}

\answerbox{q3}{3cm}{}

\item Compute the value of $E_F$ at $T=0$. \emph{(20 pt)} Justify your answer. \emph{(15 pt)}

\answerbox{q4}{3cm}{}

\item Compute the value of $E_F$ when $N_A = 0$ and $T=0$ (while $N_D \neq 0$). \emph{(10 pt)} Justify your answer. \emph{(10 pt)} \emph{Hint: in this limit the donor band plays a role similar to the valence band.}

\answerbox{q5}{3cm}{}

\begin{picture}(3,3)

\end{picture}

\een
In the questions (b), (c), and (d) use the charge conservation to derive the answers. Reminder: the concentration of conduction electrons in the Boltzmann limit is $n_e = N_C \exp[-(E_G-E_F)/kT]$, where $N_C \sim (m_e T)^{3/2}$ (and a similar expression holds for holes).

\een
\end{exam}
\end{document}


%%% Local Variables:
%%% mode: latex
%%% TeX-master: t
%%% End:
